% !TEX root = ../SU2-Scitech15.tex

\section{Physical Problem and Numerical Tools}

In this section, the main numerical algorithms of SU2 will be described with a particular emphasis on the methods used to produce the results in this paper. The convective and viscous fluxes were calculated using a second-order Roe scheme. The S-A turbulence model was used, with variables convected using a second-order upwind method. Venkatakrishnan's limiter is used. Implicit local time-stepping is used to converge the problem to a steady-state solution, and the linear system is solved using the GMRES method with a maximum error tolerance of $\mathcal{O}(10^{-10})$ for each nonlinear iteration of the flow solver. LU-SGS preconditioning was used.

\begin{wraptable}{r}{5.5cm}
  \begin{center}
  \begin{tabular}{||c|c||} \hline
    $M_\infty$ & $0.15$ \\ \hline 
    $Re_c$     & $6M$ \\ \hline
    $T_\infty$ & $300K$ \\ \hline
    $\alpha$ & $0^{\circ}$, $10^{\circ}$, $15^{\circ}$ \\ \hline
    $\mathcal{R}_{\tilde\nu}(U)$ & SA \\ \hline
  \end{tabular}
  \caption{NACA 0012 free-stream conditions.} \label{tab:naca0012_runconditions}
  \end{center}
\end{wraptable} 

The limiter uses a tunable coefficient and a characteristic length. In this work there values of the limiter coefficient were tested, 0.1, 1.0 and 10.0, along with variations in the CFL number to find appropriate values for each mesh. The limiter characteristic distance was chosen as 0.1 for the mesh with 257 points on the surface based on the recommendations of Venkatakrishan\cite{Venkatakrishnan:1993}. This value was scaled for the other meshes based on the number of surface points. The characteristic distance using this method can be found using $\Delta x_{char} = 25.7 c  / n_p $, where $n_p$ is the number of points on the surface and $c$ is the chord length.
A summary of the free-stream conditions used to model this problem are shown in Table~\ref{tab:naca0012_runconditions}. 



% !TEX root = ../SU2-Scitech15.tex

\subsection*{Reynolds-averaged Navier-Stokes (RANS) equations}
Describe the way the compressible RANS equations are being modeled 

\subsection*{Spalart-Allmaras Turbulence model}
Describe the way how the SA modeled is modeled


% !TEX root = ../SU2-Scitech15.tex

\subsection*{Numerical methods}
Describe the numerical method used for the adventive fluxes, the viscous fluxes and the turbulence model. Highlight that all are 2nd order. Describe the time stepping scheme\cite{Palacios:2014,PalaciosEconomon:2014}.
Quick description on how the limiters work and how the limiter characteristic distance has been chosen and scaled to fit for all the mesh sizes \cite{Venkatakrishnan:1993}.
% !TEX root = ../SU2-Scitech15.tex

\subsection*{Linear Solver}
Describe how the Generalized minimum residual (GMRES) works and the limits at which we are using it for this cases. Describe the use of the Lower-Upper Symmetric-Gauss-Seidel (LU-SGS).
%% !TEX root = ../SU2-Scitech15.tex

\subsection*{Convergence Optimization}
Describe the process used to find the CFL number that will converge the solution in the least number of iterations, and how this is done for each of the limiters values.
% !TEX root = ../SU2-Scitech15.tex


\section{Computational Domain}

For this paper, the simulations are run on a family of five structured meshes ranging from extremely coarse to fine. These are stated in table ~\ref{tab:naca0012_mesh}. As described in the AIAA Turbulence Model Resources online, the meshes are all C-meshes with the farfield located at a distance of 500 chord lengths from the airfoil.  The mesh spacing near the boundary in the wall normal direction meets the $y+<=1$ condition.

\begin{wraptable}{r}{5.5cm}
  \begin{center}
  \begin{tabular}{||c|c||} \hline
    $Mesh No.$ &  Mesh points\\ \hline \hline
    $G0$ &  113 X 33\\ \hline \hline
    $G1$ & 225 X 65 \\ \hline 
    $G2$     & 449 X 129 \\ \hline
    $G3$ & 897 X 257 \\ \hline
    $G4$ & 1793 X 513 \\ \hline
  \end{tabular}
  \caption{NACA 0012 meshs.} \label{tab:naca0012_mesh}
  \end{center}
\end{wraptable} 


Characteristic based farfield boundary conditions are applied to the outer boundary of the mesh and an adiabatic wall boundary condition is applied to the airfoil surface.




