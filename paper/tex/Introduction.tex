% !TEX root = ../SU2-Scitech15.tex

\section{Introduction}
One of the advantages of using an implicit solver is the ability to choose higher Courant–Friedrichs–Lewy (CFL) numbers while maintaining stability. This feature allows for the use of local time stepping which translates into a faster rate of convergence by allowing each one of the cells to evolve at its maximum time stepping rate independently of the heterogeneity of the computational domain. This capability is particularly useful when dealing with Navier-Stokes simulations on which the cells used to model the boundary layer are significantly smaller than the cells representing the farfield. 

The most common initial condition when starting a fluid flow simulation is a uniform velocity distribution, and although it seams to make sense for cells away from the object of interest, it causes the generation of sharp gradients when significantly different size grids are located together and as well as within areas neighboring no-slip boundary condition point modeling walls. The non-physical transient collection of sharp gradients can cause the solver to diverge and the use of limiters are required to preserve stability. Ideally, in a well behaved subsonic problem the limiter will be active during the transient and will decrease its effect as the flow reached steady state~\cite{Venkatakrishnan:1993}.

For this study the NACA-0012 has been used, in addition to the wealth of experimental data that is available for this geometry, it is a canonical case in the development of numerical tools for the simulation of compressible flows, which makes it an important geometry to understand the limiter effects on the flow solution. To account for the effect that the grid resolution introduces, five different C-meshes have been used ranging from 3704 points (65 points in the airfoil surface) to 919424 points (1025 points in the airfoil surface). Particular emphasis has been put in the understanding of limiters with coarse meshes, in an effort to improve accuracy and reduce computational time reducing the cost of shape design optimization.

To be able to provide a background on the tools used within the SU$^2$ suite for this analysis, the paper has been divided as follows: