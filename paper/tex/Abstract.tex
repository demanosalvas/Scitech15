% !TEX root = ../SU2-Scitech15.tex
\begin{abstract}
This article explores the capabilities and performance of the Stanford University Unstructured (SU$^2$) suite for the simulation of compressible turbulent flows past one of the most studied cases in the Aerospace community, the NACA-0012. The two-dimensional NACA 0012 is simulated at a 0.15 Mach number and 6 million Reynolds number. The flow is modeled by the Roe 2nd-order upwind scheme and the turbulence by using 2nd-order Spalart-Allmaras (SA) model. 

The focus of this study is the understanding of the effects of the Venkatakrishnan limiter in the flow solution and integrated force coefficients, as well as the role it plays on the impulsive start of fluid flow simulations. This paper provides guidelines for the choosing of length scales and limiter values, which define the behavior of this limiter mechanism. This information is particularly relevant when using coarser meshes for design purposes since the limiters effect in the flow solution is significant and becomes less pronounced as the mesh resolution increases.

SU$^2$ is an open-source (Lesser General Public License, version 2.1), integrated analysis and design tool for solving multi-disciplinary problems governed by partial differential equations (PDEs) on general, unstructured meshes.  As such, SU$^2$ is able to handle arbitrarily complex geometries, mesh adaptation, and a variety of physical problems. At its core, the software suite is a collection of C++ modules embedded within a Python framework that are built specifically for both PDE analysis and PDE-constrained optimization, including surface gradient computations using the continuous adjoint technique. 

\end{abstract}
