% !TEX root = ../SU2-Scitech15.tex
\begin{abstract}
This article explores the performance limits of the Stanford University Unstructured (SU$^2$) suite for the simulation of compressible turbulent flows past one of the most studied cases in the Aerospace community, the NACA 0012. This analysis has been performed using an upwind second order numerical scheme with the Venkatakrishnan limiter and the Spalart-Allmaras (SA) turbulence model. To find the combination of parameters that minimizes the number of iterations required to achieve convergence is of the utmost importance. In order to facilitate this task, two of the most used parameter have been selected and varied through a series of cases to better understand the effect that this have not only on the required iterations but also on the solution accuracy. The chosen parameters are the limiter value and the CFL number. For the purpose of this study, three different limiter values have been used and an optimization procedure has been performed to find the ideal CFL to find the converged solution in the minimun number of iterations.

SU$^2$ is an open-source (Lesser General Public License, version 2.1), integrated analysis and design tool for solving multi-disciplinary problems governed by partial differential equations (PDEs) on general, unstructured meshes.  As such, SU$^2$ is able to handle arbitrarily complex geometries, mesh adaptation, and a variety of physical problems. At its core, the software suite is a collection of C++ modules embedded within a Python framework that are built specifically for both PDE analysis and PDE-constrained optimization, including surface gradient computations using the continuous adjoint technique. 

\end{abstract}
