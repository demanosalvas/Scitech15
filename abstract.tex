% !TEX root = ./SU2-Scitech15.tex
\begin{abstract}

This paper discusses the current capabilities of the Stanford University Unstructured (SU$^2$) software suite for performing high-fidelity analysis and design optimization of compressible turbulent flows.  The software architecture supports rapid implementation of additional governing equation sets and a discussion of this architecture is presented in the context of turbulence modeling.  Verification and validation studies of two- and three-dimensional problems are presented including an assessment of the computational efficiency of the solver and demonstrations of the achieved order of accuracy of the numerical schemes.  The availability of inexpensive gradient information is also demonstrated via the solution of the adjoint equations for turbulent flows and the accuracy of the gradient information is validated against finite-difference results.

SU$^2$ is an open-source integrated analysis and design tool for solving complex, multi-disciplinary problems on unstructured computational domains.  At its core, the software suite is a collection of C++ modules, linked in a Python structure that performs the discretization and solution of partial differential equation-based (PDE) problems and for performing PDE-constrained optimization tasks.  The object-oriented architecture is easy to read, learn, and modify to treat unique problems across a wide range of engineering disciplines and the solver has already been extended to treat potential flow, electrodynamics, and chemically reactive gas mixture problems.  Furthermore, the software architecture supports simultaneous, tightly-coupled solutions of multiple governing equations sets, enabling complex multi-disciplinary analysis such as aerostructural, aeroacoustic, aerothermodynamic, and magnetohydrodynamic problems of interest to the aerospace community.  These capabilities and the open-source philosophy position SU$^2$ uniquely within the CFD community to become a test-bed for both novices and seasoned analysts seeking to perform high-fidelity analysis and design of complex engineering systems.

% Our vision of the future of analysis and how it relates to Vision2030 notes:
% - Predict lift/drag/moment/stability derivative/efficiency/performance/noise/emissions characteristics with certifiable accuracy over the complete flight envelope. Certifiable accuracy refers to the quality of the numerical results that would be acceptable for product certification (such as FAA certification of engine systems for icing and bird strike).
% - Simulate steady-state and time-dependent flows including problems with dynamically deforming geometries and relative body motion including possible changes in topology (e.g., real-time high-lift system deployment, aeroelastic wing response, rotor/airframe interaction, store separation).
% - Be applicable to all Mach number ranges from subsonic to hypersonic flows, from low to high Reynolds numbers.
% - Routinely simulate flows with smooth body separation, massive separation and other complex flow physics including chemically reacting flows
% - Routinely model laminar to turbulent flow transition of all modalities (T-S waves, cross-flow and Gortler instabilities; natural and bypass
% - Enable quantification of various error sources including discretization (both spatial and temporal), algebraic and modeling errors
% - Provide automated capability for simulating to overall error tolerances
% - Provide (as standard output) full quantification of numerical errors, sensitivity information, and computational uncertainty for specified  quantities.

% Vision for future CFD codes?
% High-fidelity integrated analysis and design environment
% Enable tightly coupled multi-disciplinary (and multi-fidelity?) analysis
% Quantifiable error estimation (from numerics and also from aleatory and epistemic sources)
% Validity spanning the low mach number (incompressible) to the high-hypersonic flight regimes.
% Sensitivity information
% SU2 is designed with these criteria in mind, with an eye on 


\end{abstract}